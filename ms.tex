\documentclass{article} % For LaTeX2e
\usepackage{nips13submit_e, times}
\usepackage{hyperref}
\usepackage{url}
%\documentstyle[nips13submit_09,times,art10]{article} % For LaTeX 2.09


\pdfoutput=1


\bibliographystyle{naturemag}


\title{A Filtering Technique to Improve a Noisy Event List:\\
Gaussian-Process Regression Applied to PPG-Identified Heartbeats}


\author{
Zachary Nichols\thanks{Footnote?} \\
Department of Data Science\\
Spotify University\\
New York, NY  10001 \\
\texttt{zack@nichols.com} \\
\And
David S. Spiegel \\
R\&D Data Jedi \\
Neural Net Processor: A Learning Computer \\
\texttt{dddave@gmail.com} \\
\AND
Amir Hajian \\
Canadian Institute for Theoretical Astrophysics \\
Toronto, CA \\
\texttt{ahajian@cita.utoronto.ca} \\
}

% The \author macro works with any number of authors. There are two commands
% used to separate the names and addresses of multiple authors: \And and \AND.
%
% Using \And between authors leaves it to \LaTeX{} to determine where to break
% the lines. Using \AND forces a linebreak at that point. So, if \LaTeX{}
% puts 3 of 4 authors names on the first line, and the last on the second
% line, try using \AND instead of \And before the third author name.

\newcommand{\fix}{\marginpar{FIX}}
\newcommand{\new}{\marginpar{NEW}}

% DSS comment: re-comment this to reintroduce line numbering.
\nipsfinalcopy % Uncomment for camera-ready version

\begin{document}


\maketitle

\begin{abstract}
Sometimes lists of event times have errors.
Sometimes, those errors can be reduced via Gaussian-process regression.
Identification of heart beats in wrist photoplethysmography data is one such situation.
And we did this.
And here we tell you how.
\end{abstract}

\section{Introduction}
\label{sec:Introduction}

This is an introduction.


\section{Gaussian Processes}
\label{sec:gp}

This is another section.
Gaussian process stuff is described by \cite{rasmussen2006}

Sample figure:
\begin{figure}[h]
\begin{center}
%\framebox[4.0in]{$\;$}
\fbox{\rule[-.5cm]{0cm}{4cm} \rule[-.5cm]{4cm}{0cm}}
\end{center}
\caption{Sample figure caption.}
\end{figure}

Sample table:
\begin{table}[t]
\caption{Sample table title}
\label{sample-table}
\begin{center}
\begin{tabular}{ll}
\multicolumn{1}{c}{\bf PART}  &\multicolumn{1}{c}{\bf DESCRIPTION}
\\ \hline \\
Dendrite         &Input terminal \\
Axon             &Output terminal \\
Soma             &Cell body (contains cell nucleus) \\
\end{tabular}
\end{center}
\end{table}



\subsubsection*{Acknowledgments}

We thank the Scripps Institute for data.
We thank David Steuerman for useful guidance and conversations.
We thank the team at Sum Labs for being the best colleagues a kid could want.


\bibliography{biblio.bib}

\end{document}
